\chapter*{\hfill{\centering ВВЕДЕНИЕ}\hfill}
\addcontentsline{toc}{chapter}{ВВЕДЕНИЕ}

Развитие идей реляционной алгебры Э. Кодда привело к созданию языка SQL, получившее распространение во всем мире \cite{sql_history}. Создавая сложные запросы, разработчикам требуется грамотно выстраивать логику их выполнения, используя различные соединения или подзапросы. Как результат, скорость разрабатывания программного продукта снижается.

В 1970-е годы популярность приобретает первая реализация Prolog, за основу которой взята математическая логика \cite{prolog_history}. Повышение уровня лингвистической формы записи знаний, приближенной к естественному языку, привлекло ученых и исследователей в эту область. 

С момента основания логическое программирование было признано парадигмой с наибольшим потенциалом автоматизированного использования параллелизма. Р. Ковальски в своих работах выделяет распараллеливание как сильную сторону автоматизации исчисления предикатов \cite{kowalski}. Таким образом, совмещение этих двух идей позволило бы снизить время на создание продукта, что, безусловно требует профессиональных навыков.

Цель работы -- нахождение способа ускорения выполнения запросов в базе данных на основе распараллеливания с использованием логического языка программирования. Для достижения поставленной цели необходимо решить следующий набор задач:
\begin{itemize}
	\item [--] рассмотреть подходы к представлению реляционных баз данных;
	\item [--] описать методы параллелизма плана запроса и логического запроса;
	\item [--] выполнить анализ существующих СУБД и реализаций Prolog, способных к распараллеливанию;
	\item [--] реализовать метод ускорения выполнения запросов.
\end{itemize}