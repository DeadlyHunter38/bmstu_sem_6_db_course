\chapter*{\hfill{\centering ЗАКЛЮЧЕНИЕ}\hfill}
\addcontentsline{toc}{chapter}{ЗАКЛЮЧЕНИЕ}

Задача лингвистического упрощения записи программ, приближенного к естественному языку, приводит ученых и исследователей к созданию новых средств реализаций на основе математического аппарата. Подтверждением этого является интерес и развитие идей автоматизация логического вывода.

Параллельная обработка запросов позволяет сократить время отклика системы для выполняемых запросов. В сочетании с механизмами реализаций Prolog удается ускорить получение результатов в среднем в 1,5 раза.

Цель работы курсовой работы достигнута -- найден способ ускорения выполнения запросов в базе данных на основе распараллеливания с использованием логического языка программирования. В процессе достижения поставленной цели были решены следующие задачи:
\begin{itemize}
    \item [--] рассмотрены подходы к представлению реляционных баз данных;
    \item [--] описаны методы параллелизма плана запроса и логического запроса;
    \item [--] выполнен анализ существующих СУБД, учебных БД и реализаций  Prolog, способных к распараллеливанию; обоснован их выбор;
    \item [--] реализован метод ускорения выполнения запросов.
\end{itemize}

Развитие архитектуры графических процессоров создает новые подходы
в исследовании данного направления. Использование большего числа ядер
по сравнению с центральным процессором позволит ускорить выполнения
поиска решений для логического вывода. Представление результатов эксперимента по производительности выполнения реализации GPU-Datalog \cite{gpu_datalog} перспективное направление исследования в области логики.