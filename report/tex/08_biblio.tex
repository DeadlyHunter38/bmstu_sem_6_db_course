\addcontentsline{toc}{chapter}{СПИСОК ИСПОЛЬЗОВАННЫХ ИСТОЧНИКОВ}
\renewcommand\bibname{СПИСОК ИСПОЛЬЗОВАННЫХ ИСТОЧНИКОВ}
\bibliographystyle{utf8gost705u}  % стилевой файл для оформления по ГОСТу
\begin{thebibliography}{3}
	\makeatletter
	\def\@biblabel#1{#1. }
	
	\bibitem{sql_history}
	Chamberlin D. D. Early history of SQL //IEEE Annals of the History of Computing. – 2012. – Т. 34. – №. 4. – С. 78-82.
	
	\bibitem{prolog_history}
	Koerner P. et al. Fifty Years of Prolog and Beyond //Theory and Practice of Logic Programming. – 2022. – С. 1-83.
	
	\bibitem{kowalski}
	Kowalski R. Logic for Problem Solving. Elseiver North Holland //Inc. – 1979.
	
	\bibitem{dbms_approaches}
	Gallaire H., Minker J., Nicolas J. M. Logic and databases: A deductive approach //Readings in artificial intelligence and databases. – Morgan Kaufmann, 1989. – С. 231-247.
	
	\bibitem{approach_kowalski}
	Kowalski R. Logic as a database language //on Proc. of the third British national conference on databases (BNCOD3). – 1984. – С. 103-132.
	
	\bibitem{approach_reiter}
	Reiter R. On Conceptual Modelling, chapter Towards a Logical Reconstruction of Relational Database Theory //On Conceptual Modelling. – 1984. – С. 191-233.
	
	\bibitem{logica}
	Logica [Электронный ресурс]. URL: \url{https://opensource.googleblog.com\\ /2021/04/logica-organizing-your-data-queries.html} (дата обращения: 16.05.2022).	
	
	\bibitem{stages_sql_query}
	Пантилимонов М. В., Бучацкий Р. А., Жуйков Р. А. Кэширование машинного кода в динамическом компиляторе SQL-запросов для СУБД PostgreSQL //Труды Института системного программирования РАН. – 2020. – Т. 32. – №. 1. – С. 205-220.
	
	\bibitem{source_code_postgresql}
	PostgreSQL Source Code [Электронный ресурс]. URL: https://doxygen.postgresql.org/postgres\_8c.html\#a7908e75bd9f9494fdb
	8c4b47f01a9de9 (дата обращения: 11.04.2022).
	
	\bibitem{help_understanding_code}
	How Postgres Chooses Which Index To Use For A Query [Электронный ресурс]. URL: https://pganalyze.com/blog/how-postgres-chooses-index (дата обращения: 22.04.2022).
	
	\bibitem{plan_query_postgres}
	The Internals of PostgreSQL : Chapter 3 Query Processing [Электронный ресурс]. URL: https://www.interdb.jp/pg/pgsql03.html (дата обращения: 13.04.2022).
	
	\bibitem{generetic_algorithm}
	PostgreSQL: Documentation: 14: 60.2. Genetic Algorithms [Электронный ресурс]. URL: https://www.postgresql.org/docs/current/geqo-intro2.html (дата обращения: 13.04.2022).
	
	\bibitem{postgres_explain}
	Postgres Pro Standard. Использование EXPLAIN [Электронный ресурс]. URL:
	https://postgrespro.ru/docs/postgrespro/9.5/using-explain?lang=ru (дата обращения: 02.05.2022)
	
	\bibitem{logic_approach}
	Чистяков М. Ю. Логическое программирование как одна из парадигм программирования //52-яНАУЧНАЯ КОНФЕРЕНЦИЯ. – 2016. – С. 146.
	
	\bibitem{compare_relations_and_predicates_keeping}
	Maier F. et al. PROLOG/RDBMS integration in the NED intelligent information system //Lecture Notes in Computer Science. – 2002. – С. 528-528.
	
	\bibitem{parallel_query}
	Parallel Execution [Электронный ресурс]. URL:  \url{https://www.sdcc.bnl.gov/phobos/Detectors/Computing/Orant/doc/ 
	database.804/a58227/ch\_paral.htm} (дата обращения: 04.06.2022).
	
	\bibitem{parallel_nodes_1}
	Deepak S. et al. Query processing and optimization of parallel database system in multi processor environments //2012 Sixth Asia Modelling Symposium. – IEEE, 2012. – С. 191-194.
	
	\bibitem{and_or_parallelism}
	Gupta G., Costa V. S. Cuts and side-effects in and-or parallel prolog //The Journal of logic programming. – 1996. – Т. 27. – №. 1. – С. 45-71.
	
	\bibitem{rl_similiraty}
	Wojnicki I. A Rule-based Inference Engine Extending Knowledge Processing Capabilities of Relational Database Management Systems : дис. – Ph. D. Thesis), AGH University of Science and Technology, 2004.
	
	\bibitem{comparative_db}
	DB-Engines Ranking [Электронный ресурс]. URL: \url{https://db-engines.com/en/ranking}
	(дата обращения: 21.03.2022).
	
	\bibitem{postgresql}
	PostgreSQL 14.2 Documentation [Электронный ресурс]. URL: \url{https://www.postgresql.org/docs/14/index.html} (дата обращения: 20.03.2022).
	
	\bibitem{postgres_stability}
	PostgreSQL Agility vs. Stability [Электронный ресурс]. URL: \url{https://www.enterprisedb.com/blog/postgresql-agility-vs-stability} (дата обращения: 23.06.2022).
	
	\bibitem{adventureworks}
	Adventure Works Cycles Business Scenarios [Электронный ресурс]. URL: \url{https://docs.microsoft.com/en-us/previous-versions/sql/sql-server-2008/ms124825(v=sql.100)} (дата обращения: 28.03.2022).
	
	\bibitem{northwind}
	Northwind Database [Электронный ресурс]. URL: \url{https://github.com/pthom/northwind\_psql} (дата обращения: 28.03.2022).
	
	\bibitem{chinook}
	Chinook Database [Электронный ресурс]. URL: https://github.com/lerocha/chinook-database (дата обращения: 28.03.2022).
	
	\bibitem{swi_prolog}
	SWI-Prolog documentation [Электронный ресурс]. URL: \url{https://www.swi-prolog.org/pldoc/doc\_for?object=root} (дата обращения: 17.06.2022).
	
	\bibitem{docker}
	Docker overview | Docker Documentation [Электронный ресурс]. URL: \url{https://docs.docker.com/get-started/overview/} (дата обращения: 08.05.2022).	
	
	\bibitem{ubuntu}
	Official Ubuntu Documentation [Электронный ресурс]. URL: 
	\url{https://help.ubuntu.com/lts/ubuntu-help/index.html} (дата обращения 08.05.2022).
	
	\bibitem{explain_analyze}
	PostgreSQL Documentation: 14: Using EXPLAIN [Электронный ресурс]. URL:  \url{https://www.postgresql.org/docs/current/using-explain.html} (дата обращения: 18.06.2022).
	
	\bibitem{gpu_datalog}
	Abiteboul S., Vianu V. Datalog extensions for database queries and updates //Journal of Computer and System Sciences. – 1991. - Т. 43. – №. 1. – С. 62-124.
	
\end{thebibliography}
