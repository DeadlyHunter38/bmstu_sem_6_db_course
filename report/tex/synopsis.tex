\chapter*{\hfill{\centering РЕФЕРАТ}\hfill}

Расчетно-пояснительная записка \pageref{LastPage} с., \totalfigures\ рис., \totaltables\ табл., 30 ист., 2 прил.

Ключевые слова: \textit{Базы Данных}, \textit{SQL}, \textit{Prolog}, \textit{Исчисление предикатов}, \textit{Параллелизм}, \textit{Логический вывод}.

Объектом исследования является метод ускорения запросов в базе данных.

Цель работы -- нахождение способа ускорения выполнения запросов в базе данных на основе распараллеливания с использованием логического языка программирования. Для достижения поставленной цели необходимо решить следующий набор задач:
\begin{itemize}
	\item [--] рассмотреть подходы к представлению реляционных баз данных;
	\item [--] описать методы параллелизма плана запроса и логического запроса;
	\item [--] выполнить анализ существующих СУБД и реализаций Prolog, способных к распараллеливанию;
	\item [--] реализовать метод ускорения выполнения запросов.
\end{itemize}

В результате выполнения работы был найден способ ускорения выполнения запросов в базе данных на основе распараллеливания с использованием логического языка программирования. 

Параллельная обработка запросов позволяет сократить время отклика системы для выполняемых запросов. В сочетании с механизмами реализаций Prolog удается ускорить получение результатов в среднем в 1,5 раза.

Развитие архитектуры графических процессоров создает новые подходы
в исследовании данного направления. Использование большего числа ядер
по сравнению с центральным процессором позволит ускорить выполнение
поиска решений для логического вывода.