\usepackage{cmap}

\usepackage[utf8]{inputenc}
\usepackage[english,russian]{babel}

\usepackage[left=3cm, right=2cm, top=2cm, bottom=2cm]{geometry}
\usepackage{setspace}
\onehalfspacing % Полуторный интервал

\usepackage[shortlabels]{enumitem}
\usepackage{amsmath} %фигурная скобка
\usepackage{autonum}
\usepackage{multicol, multirow}
\usepackage{csvsimple}
\usepackage{graphicx}
\graphicspath{{noiseimages/}}

\usepackage{caption}
\captionsetup{labelsep=endash}
\captionsetup[figure]{name={Рисунок}}

\usepackage{indentfirst} % Красная строка

\usepackage{float}
\usepackage{umoline}

% Пакет Tikz
\usepackage{tikz}
\usetikzlibrary{arrows,positioning,shadows}
\usepackage{pgfplots, pgfplotstable}
\usepackage{pdfpages}
\setcounter{page}{3}

%url inside pdf
\usepackage[hidelinks]{hyperref}
\hypersetup{urlbordercolor=white,
			colorlinks=false}

\usepackage{xcolor}
\definecolor{darkgray}{gray}{0.15}

%rotate figure and lanel
\usepackage{rotating}
\usepackage{epsfig}
\usepackage{subfig}

%код
\usepackage{listings} 
\captionsetup[lstlisting]{singlelinecheck = false, justification=raggedright}
\lstset{
	basicstyle=\footnotesize\ttfamily,
	language=[Sharp]C, % Или другой ваш язык -- см. документацию пакета
	commentstyle=\color{comment},
	numbers=left,
	numberstyle=\tiny\color{plain},
	numbersep=5pt,
	tabsize=4,
	extendedchars=\true,
	breaklines=true,
	keywordstyle=\color{blue},
	frame=b,
	stringstyle=\ttfamily\color{string}\ttfamily,
	showspaces=false,
	showtabs=false,
	xleftmargin=17pt,
	framexleftmargin=17pt,
	framexrightmargin=5pt,
	framexbottommargin=4pt,
	showstringspaces=false,
	inputencoding=utf8x,
	keepspaces=true
}

\DeclareCaptionLabelSeparator{line}{\ --\ }
\DeclareCaptionFont{white}{\color{white}}
\DeclareCaptionFormat{listing}{\colorbox[cmyk]{0.43,0.35,0.35,0.01}{\parbox{\textwidth}{\hspace{15pt}#1#2#3}}}
\captionsetup[lstlisting]{
	format=listing,
	labelfont=white,
	textfont=white,
	singlelinecheck=false,
	margin=0pt,
	font={bf,footnotesize},
	labelsep=line
}

\frenchspacing

\usepackage{titlesec}
%section
\titleformat{\section}[block]
{\bfseries\normalsize\filcenter}{\thesection}{1em}{}

%subsection
\titleformat{\subsection}[hang]
{\bfseries\normalsize}{\thesubsection}{1em}{}
\titlespacing\subsection{\parindent}{\parskip}{\parskip}

%subsubsection
\titleformat{\subsubsection}[hang]
{\bfseries\normalsize}{\thesubsubsection}{1em}{}
\titlespacing\subsubsection{\parindent}{\parskip}{\parskip}

% Подсчет изображений, таблиц
\usepackage[figure,table]{totalcount} 

% Поворот изображения вместе с названием
\usepackage{rotating}

% Работа с изображениями и таблицами; переопределение названий по ГОСТу
\usepackage{caption}
\captionsetup[figure]{name={Рисунок},labelsep=endash}
\captionsetup[table]{singlelinecheck=false, labelsep=endash}

% Для подсчета числа страниц
\usepackage{lastpage}

\titlespacing*{\chapter}{0pt}{-30pt}{8pt}
\titlespacing*{\section}{\parindent}{*4}{*4}
\titlespacing*{\subsection}{\parindent}{*4}{*4}

\newcommand{\hsp}{\hspace{20pt}} % длина линии в 20pt

\titleformat{\chapter}[hang]{\Huge}{\thechapter\hsp}{0pt}{\Huge\textmd}

\titleformat{\section}{\Large}{\thesection}{18pt}{\Large\textmd}
\titleformat{\subsection}{\Large}{\thesubsection}{16pt}{\Large\textmd}
\titleformat{\subsubsection}{\normalfont\textmd}{}{0pt}{}

\linespread{1.25} %межстрочный интервал
\newcommand{\anonsection}[1]{ \section*{#1} \addcontentsline{toc}{section}{\numberline {}#1}}

\makeatletter
\newenvironment{sqcases}{%
	\matrix@check\sqcases\env@sqcases
}{%
	\endarray\right.%
}
\def\env@sqcases{%
	\let\@ifnextchar\new@ifnextchar
	\left\lbrack
	\def\arraystretch{1.2}%
	\array{@{}l@{\quad}l@{}}%
}
\makeatother
